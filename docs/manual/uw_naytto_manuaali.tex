\documentclass[12pt,a4paper,finnish]{uvmanual}
\special{papersize=210mm,297mm}


\title{Usewood kosketusnäytön manuaali}


\renewcommand{\labelitemii}{}


\begin{document}


%	
% Create the title page.
% First the logo. Check its language.
\thispagestyle{empty}
%\vspace*{-.5cm}\noindent
\vspace*{-1cm}\noindent
\begin{center}
\includegraphics[width=8cm]{img/uw_logo_color.png}   % Bilingual logo
\end{center}
\vspace{1cm}
\begin{center}
\includegraphics[width=8cm]{img/Usevolt.png}   % Bilingual logo
\end{center}



% Then lay out the author, title and type to the center of page.
\vspace{2.8cm}
\maketitle
\vspace{2.8cm}


% Last some additional info to the bottom-right corner
\begin{minipage}[c]{8.8cm}
  \begin{spacing}{1.0}
    \textsf{Usewood Forest Tec Oy}\\
    \textsf{www.usewood.fi}\\
    \textsf{Usevolt Oy}\\
    \textsf{www.usevolt.fi}\\
  \end{spacing}
\end{minipage}

% Leave the backside of title page empty in twoside mode
\if@twoside
\clearpage
\fi

\pagenumbering{roman} 
\setcounter{page}{0} % Start numbering from zero because command 'chapter*' does page break


% Some fields in abstract are automated, namely those with \@ (author,
% title in the main language, thesis type, examiner).



% Add the table of contents, optionally also the lists of figures,
% tables and codes.


\renewcommand\contentsname{Sisällysluettelo}         % Set Finnish name
\setcounter{tocdepth}{2}                      % How many header level are included
\tableofcontents                              % Create TOC

% Leave the backside of abbreviation list empty in twoside mode
\cleardoublepage

% The actual text begins here and page numbering changes to 1,2...
\newpage             % needed for page numbering
\pagenumbering{arabic}
\setcounter{page}{1} % Start numbering from zero because command
                     % 'chapter*' does page break
\renewcommand{\chaptername}{} % This disables the prefix 'Chapter' or
                              % 'Luku' in page headers (in 'twoside'
                              % mode)

\chapter{Johdanto}

Tämä dokumentti on käyttöohje Usewood pienmetsäkoneiden näytön käyttöliittymään. Näytön pääasiallinen käyttötarkoitus on toimia käyttöliittymänä koneen parametrien säätämiseen ja ajon aikaisten tietojen näyttämiseen kuljettajalle. Monet koneen parametreista pystytään asettamaan vain näytön kautta, joten kuljettajalta vaaditaan sen käytön sujuvaa osaamista. Koneen säätämättä jättäminen saattaa tehdä koneella työskentelystä vaikeaa ja jopa altistaa vaaratilanteille työskenneltäessä.

\warning{Yleinen varoitussymboli}
{Varoitussymboleja käytetään tässä manuaalissa korostamaan kohtia, jotka ovat tärkeitä koneen turvallisuuden kannalta. Lue varoitussymboleiden kuvaukset noudattaen erityistä huolellisuutta.}

\chapter{Tietopaneli}\label{ch:taskbar}
Tietopaneli on aina ruudun alalaidassa näkyvä paneli, joka sisältää tärkeimmät käytönaikaiset tiedot Usewood pienmetsäkoneesta. Alalaidan panelin lisäksi ruudun vasemmassa ja oikeassa laidassa olevat mittaristot ovat kokoajan näkyvillä.

\fig{img/taskbar}{Tietopaneli}{taskbar}

\begin{tabular}{ L{0.2\textwidth} L{0.66\textwidth} }
\textbf{RPM} & Moottorin kierrosluku (kierrosta minuutissa) \\
\textbf{Pressure} & Hydrauliikkapumpulta tuleva hydrauliikkapaine. \\
\textbf{Hours} & Moottorin käyttötuntilaskuri. Laskurin arvoa ei voi muuttaa, vaan se laskee moottorin käyttötunteja jatkuvasti. \\
\textbf{Horn} & Äänimerkki. Paina tästä antaaksesi varoitusäänimerkin. \\
\textbf{Bat V} & Akun jännite. Akun jännitteen päästäminen liian matalaksi saattaa vaurioittaa akkua. \\
\textbf{Fuel L} & Polttoaineen määrä. \\
\textbf{Oil L} & Hydrauliikkaöljyn määrä. \\
\textbf{Oil T} & Hydrauliikkaljyn lämpötila. \\
\textbf{Motor T} & Moottorin lämpötila. \\
\textbf{Clock} & Kellonaika. \\
\end{tabular}

\warning{Warning}{Akkujännitteen päästäminen liian matalaksi tai hydrauliikkaöljyn lämpötilan päästäminen liian korkeaksi saattaa vaurioittaa konetta pysyvästi. Käyttäjän on syytä valvoa näitä tämän näytön kautta aina konetta käytettäessä.}

\chapter{Sisäänkirjautuminen}\label{ch:login}

Sisäänkirjautumisnäyttö näytetään aina kun koneeseen laitetaan virrat päälle. Sen kautta pystytään luomaan uusia kuljettajia ja valitsemaan aktiivinen kuljettaja. Kun kuljettajaa vaihdetaan, lataa näyttö automaattisesti kyseisen kuljettajan asetukset koneeseen. Oletuskäyttäjän nimi on \textbf{Usewood} ja maksimissaan 4 kuljettajaa voidaan luoda.

\fig{img/login.png}{Sisäänkirjautuminen}{login}

\begin{tabular}{ L{0.2\textwidth} L{0.66\textwidth} }
\textbf{Log in} & Kirjaa sisään sen käyttäjän mikä on valittuna sivun vasemmassa laidassa ja lataa käyttäjän asetukset koneeseen. \\
\textbf{Add user} & Luo uuden käyttäjän. Näytölle ilmestyy näppäimistö, jonka kautta käyttäjälle on annettava nimi. \\
\textbf{Delete user} & Poistaa käyttäjän ja sen asetukset. Käyttäjää ei pystytä enää palauttamaan. Jos koneeseen on asetettu vain yksi käyttäjä, ei tätä voida poistaa. \\
\end{tabular}


\chapter{Kotinäyttö}\label{ch:home}

\fig{img/home}{Kotinäyttö}{home}

\begin{tabular}{ L{0.2\textwidth} L{0.66\textwidth} }
\textbf{Dashboard} & Näyttää valitun työlaitteen työnäytön. Erityisesti Uw180s hakkuukouraa käytettäessä \textbf{Dashboard} näyttää karsittavan puun pituuden ja paksuuden. \\
\textbf{Settings} & Näyttää käyttäjäkohtaiset asetukset. Koneen parametrien säätö tehdään tässä valikossa. \\
\textbf{System} & Näyttää järjestelmädiagnostiikkaikkunan. Järjestelmän vikadiagnosointi ja kalibrointi tehdään täällä. \\
\textbf{Log out} & Kirjaa nykyisen käyttäjän ulos ja palaa sisäänkirjautumisnäyttöön. \\
\end{tabular}



\chapter{Työnäyttö}\label{ch:dashboard}

Työnäytön sisältä riippuu valitusta työlaitteesta. Työlaite valitaan asetuksista. Työnäyttö on hyödyllinen erityisesti UW180s hakkuukouraa käytettäessä.

\fig{img/dashboard}{Työnäyttö}{dashboard}

\begin{tabular}{ L{0.2\textwidth} L{0.66\textwidth} }
\textbf{Cancel \& OK} & Palaa takaisin kotinäytölle. \\
\textbf{Log length} & Vaihtaa pöllin tavoitemittaa. Mahdolliset arvot on asetettava UW180s hakkuukouran asetuksista.\\
\textbf{Length} & Näyttää parhaillaan karsittavan puun pituuden sentimetreissä. \\
\textbf{Diameter} & Näyttää parhaillaan karsittavan puun paksuuden millimetreissä. \\
\textbf{Volume} & Näyttää hakattujen pöllien yhteenlasketun tilavuuden. Tilavuus voidaan nollata UW180s hakkuukouran asetuksista. \\
\end{tabular}


\chapter{Asetukset}\label{ch:settings}

Asetuksista säädetään kaikki koneen käyttäjäkohtaiset asetukset. Minkä tahansa asetuksen muutos ladataan ohjausjärjestelmään automaattisesti, mutta muutokset tallennetaan pysyvästi vasta kun käyttäjä on painanut \textbf{OK} näppäintä. Painamalla \textbf{Cancel} perutaan kaikki tehdyt muutokset. Asetukset on luokiteltu seuraaviin välilehtiin:

\begin{itemize}
 \item General settings (\autoref{ch:settings_general})
 \item Valve Configurations (\autoref{ch:settings_valves})
 \item Implement settings (\autoref{ch:settings_implement})
\end{itemize}


\warning{Warning}{Vaikka konetta ei pysty rikkomaan parametrien huonoilla / väärillä asetuksilla, on käyttäjän kuitenkin mahdollista säätää parametrit niin, että kone joko toimii huonosti tai niin että sen käyttö on hankalaa. On käyttäjän vastuulla säätää parametreja huolellisesti, ehkäisten koneen suorituskyvyn heikkeneminen.}

\section{Yleiset asetukset}\label{ch:settings_general}


\fig{img/settings_general}{Yleiset asetukset}{settings_general}

\begin{tabular}{ L{0.2\textwidth} L{0.66\textwidth} }
\textbf{Screen brightness} & Asettaa näytön kirkkauden.\\
\textbf{Drive Lights} & Asettaa ajovalot päälle tai pois päältä. Ajovalot koostuvat eteenpäin osoittavista tehoLEDeistä ja taaksepäin osoittavasta punaisesta merkkivalosta. \\
\textbf{Work Lights} & Asettaa työvalot päälle tai pois päältä. Työvalot koostuvat sivuille osoittavista tehoLEDeistä. \\
\textbf{Wiper} & Asettaa tuulilasinpyyhkijän nopeuden. \\
\textbf{Heater} & Asettaa lämminilmapuhaltimen päälle tai pois päältä. \\
\end{tabular}




\section{Venttiilit}\label{ch:settings_valves}

Hydrauliikkaventtiileiden parametrien säätäminen on yksi tärkeimmistä asioista mikä jokaisen kuljettajan tulisi säätää itselleen sopivaksi, jotta paras mahdollinen ajokokemus pystytään saavuttamaan. Jokaisen käyttäjän on mahdollista säätää jokaisen liikkeen säädöt juuri itselleen sopiviksi.


\warning{Warning}{Hyvä ja huolellinen venttiiliasetusten säätö parantaa ajokokemusta huomattavasti. Huonosti säädetyt asetukset taas voivat tehdä koneesta jopa vaarallisen. Käytä aina harkintaa säätäessäsi venttiileiden asetuksia.}


Jokaisella venttiilillä on samat asetukset säädettävänä. Usewoodin pienmetsäkoneissa on vähintäänkin seuraavat venttiilit:

\begin{itemize}
 \item Puomin nosto
 \item Puomin taitto
 \item Puomin kääntö
 \item Ajo
 \item Ohjaus
 \item Vasen tukijalka
 \item Oikea tukijalka
\end{itemize}

Painamalla minkä tahansa venttiilin nimeä aukeaa kyseisen venttiilin parametrit asetettavaksi.

\fig{img/settings_valves}{Runkokoneen venttiiliasetukset}{settings_valves}

\fig{img/settings_valves_boom_lift}{Yksittäisen venttiilin asetukset}{settings_valves_boom_lift}

\begin{tabular}{ L{0.2\textwidth} L{0.66\textwidth} }
\textbf{Back} & Palaa takaisin venttiileiden valintanäyttöön. \\
\textbf{Boom Lift} & Tämänhetkisen venttiilin nimi, minkä asetuksia ollaan säätämässä. \\
\textbf{Invert Direction} & Venttiilin suunnan invertointi. Kaikki venttiilit ovat kaksitoimisia. Tämä vaihtaa venttiilin toimintasuunnan. \\
\textbf{Forward min speed} & Asettaa minimiohjausvirran venttiilille milliampeereissa. Kaikilla hydrauliventtiileillä on tietty virta, millä ne alkavat päästämään öljyä lävitseen. \\
\textbf{Forward max speed} & Asettaa maksimiohjausvirran venttiilille milliampeereissa. Toimii maksimiliikenopeuden säätönä. \\
\textbf{Acceleration} & Asettaa venttiilivirran kasvukiihtyvyyden. Tämä tunnetaan myös nousuramppina. Pieni arvo saa liikkeen kiihtymään rauhallisesti. \\
\textbf{Backward min speed} & Sama kuin \textbf{Forward min speed}, mutta vastakkaiseen suuntaan. \\
\textbf{Backward max speed} & Sama kuin \textbf{Forward max speed}, mutta vastakkaiseen suuntaan. \\
\textbf{Deceleration} & Asettaa venttiilivirran laskunopeuden. Tunnetaan myös laskuramppina. Pieni arvo saa venttiilivirran laskeutumaan hitaasti kohti nollaa, aiheuttaen liikkeen sulavan pysähtymisen. \\
\end{tabular}

\warning{Warning}{Minimivirrat tulee määrittää varoen. Liian suuri minimivirta saa liikkeen lähtemään nytkähtäen liikkeelle, huolimatta siitä kuinka varovasti kuljettaja ohjauskahvaa kääntäisi.}

\warning{Warning}{Hidastuvuus tulee määrittää varoen. Liian pieni hidastuvuus saa liikkeen jatkumaan vaikka kuljettaja olisi jo lopettanut ohjaamisen.}

\section{Työlaiteasetukset}\label{ch:settings_implement}

Työlaiteasetukset määrittävät valitun työlaitteen asetukset. Työlaitteen asettaminen tapahtuu \textbf{Yleiset} välilehdeltä. Tässä ohjekirjassa käsitellään UW180s hakkuukouran asetukset, sillä ne ovat kaikista monimutkaisimmat ja muiden työlaitteiden asetukset pohjautuvat niihin.

\fig{img/settings_uw180s}{Työlaitteen asetukset}{settings_implement}

\begin{tabular}{ L{0.2\textwidth} L{0.66\textwidth} }
\textbf{Log Measurement} & Pöllin mittausasetukset ja mittalaitteen kalibrointiasetukset\\
\textbf{Valve Configurations} & UW180s Venttiiliasetukset 
\end{tabular}

\FloatBarrier
\subsection{UW180s mittalaiteasetukset}\label{ch:settings_uw180s_meas}

\fig{img/settings_uw180s_meas}{UW180s measurement settings screen}{settings_uw180s_meas}

\begin{tabular}{ L{0.2\textwidth} L{0.66\textwidth} }
\textbf{Back} & Palaa takaisin UW180s asetuksiin. \\
\textbf{Log length 1} & Ensimmäinen pöllin pituus millimetreissä. \\
\textbf{Log length 2} & Toinen pöllin pituus millimetreissä. \\
\textbf{Length Calibration} & Kalibrointiarvo pituusmitalle. Mitä suurempi arvo tähän laitetaan, sitä lyhyempiä pöllejä kone tekee. \\
\textbf{Volume Calibration} & Tilavuuden laskennan korjauskerroin. \\
\end{tabular}

\FloatBarrier
\subsection{UW180s venttiiliasetukset}\label{ch:settings_uw180s_valves}

Venttiiliasetukset toimivat samoin kuin runkokoneen venttiiliasetukset.

\fig{img/settings_uw180s_valves}{UW180s venttiiliasetukset}{settings_uw180s_valves}

\begin{tabular}{ L{0.2\textwidth} L{0.66\textwidth} }
\textbf{Wheels} & Vetorullien aukaisu / sulkeminen \\
\textbf{Wheels feed} & Puun karsinta \\
\textbf{Delimbers} & Karsintaterät \\
\textbf{Saw} & Sahaus \\
\textbf{Tilt} & Hakkuukouran tilttaus / puun kaato \\
\textbf{Rotator} & Hakkuukouran pyöritys \\
\end{tabular}

\chapter{Järjestelmä}\label{ch:system}

Järjestelmävalikko on organisoitu samoin kuin asetusvalikko. \textbf{OK}-painike tallentaa asetukset pysyvästi ja \textbf{Cancel}-painike hylkää tehdyt muutokset.

\section{Asetukset}\label{ch:system_settings}

Järjestelmäasetukset ovat kaikkien käyttäjien kesken jaettuja asetuksia, jotka koskevat koko koneen toimintaa.

\fig{img/system_settings_1}{Järjestelmäasetukset 1/2}{system_settings_1}

\fig{img/system_settings_2}{Järjestelmäasetukset 2/2}{system_settings_2}

\begin{tabular}{ L{0.2\textwidth} L{0.66\textwidth} }
\textbf{Engine power usage} & Määrittelee kuinka paljon tehoa moottorilta otetaan. Jos tämä on liian pieni, toimii kone vajaatehoisena. Jos arvo taas on liian suuri, rasittuu moottori savuttaen mustaa savua ja se voi jopa sammua kesken työskentelyn. \\
\textbf{Joystick Calibration} & Ohjauskahvojen kalibrointi. Kun painike on kerran painettu, tulee ohjauskahvoja pyöräyttää kaikkien akseleidensa maksimi- ja minimiarvot läpi. Sen jälkeen painiketta uudelleen painaessa kalibrointi on suoritettu. Ohjauskahvat tarvitsee kalibroida muutaman sadan tunnin välein. \\
\textbf{Date and time} & Asettaa kellonajan ja päivämäärän. \\
\end{tabular}



\warning{Warning}{Ohjauskahvojen epäonnistunut kalibrointi aiheuttaa koneen käytettävyyden heikkenemisen. Ohjauskahvojen ylimääräinen kalibrointi on turhaa.}

\warning{Warning}{Jos kellonaika ja päivämäärä unohtuvat aina kun koneesta irroitetaan akku, on näytön sisäinen 2032-nappiparisto tyhjentynyt.}

\section{Diagnostiikka}\label{ch:system_diagnostics}

Diagnostiikkanäyttö on tarkoitettu ohjausjärjestelmän vikatilanteiden selvittämistä varten. Se näyttää kaiken ohjausjärjestelmässä kulkevan raa'an datan. Ohjausjärjestelmä koostuu seuraavista ohjaimista:

\begin{tabular}{ L{0.2\textwidth} L{0.66\textwidth} }
\textbf{ESB} & Moottorin ohjain \\
\textbf{CSB} & katon ohjain \\
\textbf{FSB} & Etukonsolin ohjain \\
\textbf{Left Keypad} & Vasen ohjauskahva ja näppäimistö \\
\textbf{Right Keypad} & Oikea ohjauskahva ja näppäimistö \\
\textbf{Pedal} & Ajopoljin \\
\textbf{HCU} & Pääventtiilin ohjain \\
\textbf{CCU} & Sivuventtiilin ohjain \\
\textbf{ICU} & UW180s hakkuukouran ohjain \\
\end{tabular}



\fig{img/system_diagnostics}{Diagnostiikkanäyttö}{system_diagnostics}

Vaikka osa näytöllä näytettävästä datasta on hyödyllistä vain vikatilanteiden ratkaisemisessa, on osa siitä myös hyödyllistä vaikkei erityistä vikaa koneessa olisikaan.

\FloatBarrier
\subsection{ESB Diagnostiikka}\label{ch:system_diagnostics_msb}

\fig{img/system_diagnostics_msb}{ESB Diagnostiikkanäyttö}{system_diagnostics_msb}

\textbf{ESB} vastaa suurimman osan tietopaneelin näyttämän tiedon ylläpidosta. Se kertoo mm. moottorin lämpötilan, kierrosluvun, ja hydrauliikkaöljyn määrän.

\FloatBarrier
\subsection{HCU diagnostics}\label{ch:system_diagnostics_ecu}

\fig{img/system_diagnostics_ecu}{HCU diagnostiikkanäyttö}{system_diagnostics_ecu}

\textbf{HCU} vastaa koneen pääventtiilin ohjauksesta. Sen ja \textbf{CCU}:n kautta on nähtävissä jokaiselle venttiilille kyseisellä hetkellä menevä ohjausvirta.



\section{Tehdasasetusten palautus}\label{ch:system_restore}

\textbf{System Restore} Palauttaa koko ohjausjärjestelmän tehdasasetukset.

\fig{img/system_restore}{Tehdasasetusten palautus}{system_restore}

\begin{tabular}{ L{0.2\textwidth} L{0.66\textwidth} }
\textbf{Restore System Defaults} & Painamalla painiketta niin kauan kunnes näytölle ilmestyvä laskuri laskee 10:stä nollaan palauttaa järjestelmän tehdasasetukset. Tehdasasetusten palutus poistaa kaikkien tallennettujen käyttäjien asetukset. \\
\end{tabular}
























\end{document}
